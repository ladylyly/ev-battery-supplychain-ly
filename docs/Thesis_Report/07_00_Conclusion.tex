\chapter{Conclusion}
\label{chapter:conclusion}
\begin{comment}
* Summarize your thesis.
* Emphasize your contribution and outcome.
* Summarize the results of your evaluation again.
* Summarize the most important aspects of your thesis.
* Discuss limitations.
* Give an outlook of future work: What would you do next if you were to keep working in this field?
* 2+ pages
\end{comment}

The goal of this thesis was to reconcile two seemingly conflicting requirements in the context of electric vehicle (EV) battery supply chains. On one hand, regulators and downstream stakeholders demand verifiable and end-to-end traceability. On the other hand, suppliers and original equipment manufacturers (OEMs) need to protect sensitive business information such as prices and transactional details from public exposure. Existing work in our group had already shown that verifiable credentials (VCs) can represent complex supply-chain transformations and that escrow-based marketplaces on Ethereum can coordinate trades without a central intermediary. However, these systems exposed prices and transaction identifiers in plaintext in smart contracts, events, or credentials, and they did not provide systematic mechanisms for selective disclosure.

This thesis set out to close that gap. Building on the prior work on VC-based traceability and marketplace designs, it developed and implemented an architecture in which supply-chain events are represented as chained VCs, and the public blockchain serves only as an anchor and coordination layer. Numeric fields such as prices and selected transaction identifiers are no longer written to contract storage or events in plaintext. Instead, they are committed to using Pedersen commitments over the Ristretto255 curve, and Bulletproof-based zero-knowledge proofs allow verifiers to check that these hidden values satisfy required constraints, for example, that they lie within a valid range or are bound to a specific product and lifecycle stage, without revealing the exact numbers. Large payloads, including VC JSON, proofs, and signatures, remain off-chain on IPFS. Contracts store only compact anchors, such as content identifiers and commitments, and emit events for lifecycle transitions.

The research was guided by four core questions. The first question asked whether a VC schema and chaining model can provide end-to-end traceability for EV battery components and products without modeling them as on-chain assets. By defining a VC structure that captures lifecycle stages, from listing to delivery, as well as provenance links between components and assembled products, and by demonstrating that an auditor can reconstruct and verify complete product histories using only VCs, signatures, and on-chain anchors, the thesis showed that such traceability is achievable in practice.

The second question focused on the extent to which a design based on Pedersen commitments and Bulletproofs can hide prices and transaction identifiers in public artefacts, while still allowing verifiers to check required constraints and bindings. The implemented commitment and binding-tag design, together with a Rust-based Bulletproof backend, ensured that no plaintext prices or transaction hashes appeared in contract storage, events, or VC JSON. Security validation tests confirmed that attempts at misbinding, replay, and tampering are rejected.

The third research question concerned the cost of this additional cryptographic privacy layer. A naive design that stores full VC payloads directly on-chain is prohibitively expensive in terms of gas and storage. The evaluation in this thesis compared such a naive baseline with the implemented anchor-only design. The results showed that keeping only commitments and content identifiers on-chain reduces storage and gas consumption by orders of magnitude, while still preserving verifiability through off-chain reconstruction. Bulletproof generation and verification times for 64-bit price ranges were measured in the tens of milliseconds on commodity hardware, which is acceptable for backend services and auditor tools.

The fourth research question asked whether auditors and other stakeholders, using only public chain data, IPFS, and the proof backend, can reconstruct product histories and verify both provenance and price-related constraints in a way that is robust against tampering. The implemented auditor workflow and the security validation tests demonstrated that this is possible. Auditors can follow VC chains, check signatures, commitments, binding tags, and proofs, and detect inconsistencies introduced by malicious modifications.

Taken together, these findings show that it is possible to combine credential-based traceability, minimal on-chain state, and cryptographic privacy mechanisms in a single coherent architecture for EV battery supply chains. The proposed design keeps the public artefacts free of sensitive price and transaction-identification data, yet it maintains the ability of independent auditors to verify both provenance and compliance with price-related policies. At the same time, the system remains deployable on an Ethereum-compatible chain, owing to the use of content-addressed storage, event-based anchoring, and the minimal proxy pattern, EIP-1167, for per-product escrow contracts.

Nonetheless, several limitations remain. Payment flows themselves are still visible on the underlying blockchain. While prices and transaction identifiers are hidden inside the traceability layer, transaction amounts and counterparties in the settlement layer can still be observed. This means that an adversary with access to on-chain payment data can potentially correlate payment patterns with operational events, even if the VCs and contracts hide price fields. Additionally, zero-knowledge proofs are generated and, in many scenarios, verified by a dedicated backend service. This introduces a dependency on that backend’s correctness and availability. Verification also relies on the continued availability of VC payloads on IPFS, which is mitigated but not eliminated by pinning. Lastly, the prototype focuses on a single chain, a limited number of products, and relatively short provenance chains. Scalability to industrial-scale deployments and comprehensive coverage of all EU battery passport attributes remain areas for future work.

Despite these limitations, this thesis contributes a concrete and implemented example of how privacy-preserving techniques can be integrated into a VC-based traceability framework without sacrificing auditability. It demonstrates that hiding prices and transaction identifiers is not only cryptographically possible but also compatible with practical constraints on gas, storage, and performance. For practitioners and researchers, the work offers a blueprint for building similar systems in other regulated domains where sensitive business data must be protected while maintaining verifiable provenance.

Possible future work includes several directions. The first avenue is the integration of fully private payment systems, such as shielded-pool constructions in the style of Railgun, so that payment amounts and counterparties are hidden on-chain while credentials still attest to the fact of settlement. This would require binding proofs about shielded transfers to specific products and credentials, and carefully evaluating the resulting privacy and performance trade-offs. Another direction is to reduce trust in the proof backend by developing browser-based or desktop verification tooling, for example, using WebAssembly, so that auditors can verify Bulletproofs locally without relying on a central service. A third direction is to conduct large-scale experiments with many products, long provenance chains, and concurrent transactions to better understand scalability and user experience under realistic workloads. Finally, extending the credential model to cover a broader set of EV battery passport attributes and integrating it with emerging industry standards and governance mechanisms would bring the prototype closer to deployment in real supply chains.

In summary, this thesis demonstrates that verifiable credentials, smart contracts, and modern zero-knowledge proof systems can be combined to provide both traceability and confidentiality in EV battery supply chains. While further research is needed to achieve fully private payments and to scale to industrial deployments, the results presented here indicate that privacy-preserving, auditable supply-chain systems on public blockchains are a realistic and promising direction.